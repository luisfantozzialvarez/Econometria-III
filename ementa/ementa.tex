\documentclass[10pt,a4paper]{article}
\usepackage[T1]{fontenc}
\usepackage{graphicx}
\usepackage{mathtools}
\usepackage{amssymb}
\usepackage{amsthm}
\usepackage{thmtools}
\usepackage{xcolor}
\usepackage{nameref}
\usepackage{hyperref}
\title{\large EAE1223: Econometria III}
\author{Professor Luis A. F. Alvarez}
\date{}
\begin{document}
	\maketitle
	\paragraph{Objetivos do curso:} o objetivo do curso é apresentar aos alunos os fundamentos estatísticos e econométricos para a análise de problemas em séries de tempo.
	\paragraph{Tópicos:} cobriremos os seguintes grandes tópicos.
	\begin{enumerate}
		\item Processos estocásticos estacionários e não estacionários.
		\item Decomposições descritivas de séries de tempo: tendência, ciclo e sazonalidade.
		\item Testes de não estacionariedade.
		\item Metodologia de Box-Jenkins para previsão univariada.
		\item Modelos autorregressivos vetoriais em forma reduzida.
		\item Cointegração e correção de erros.
		\item Inferência robusta à correlação serial.
		\item Modelos autorregressivos vetoriais estruturais.
		\item Heterocedasticidade condicional.
		\item Tópicos adicionais.
	\end{enumerate} 
	\paragraph{Estrutura do curso:} o curso consiste de duas aulas semanais e uma monitoria obrigatória. Nas aulas teóricas, o professor exporá os conteúdos e apresentará implementações computacionais, na linguagem \texttt{R}, dos métodos em aula. Nas monitorias, os alunos implementarão, em duplas, os métodos vistos em aula em atividades práticas.
	
	\paragraph{Sobre o \texttt{R}:}  os alunos devem fazer o \textit{download} e instalação do intérprete de R, disponível em \url{https://www.r-project.org/}. O intérprete é o \textit{software} que interpreta os códigos em R e executa as operações neles ditadas. Esse
	\textit{software} não precisa ser acessado diretamente. Para escrever e rodar os códigos, é mais conveniente usarmos um ambiente próprio para isso, que
	possui ferramentas de autocompletamento e visualização das variáveis. Para isso, os alunos devem baixar também a ferramenta Rstudio (\url{https://www.rstudio.com/download}). É através desse programa que escreveremos e rodaremos os códigos.
	\paragraph{Avaliação:} a nota final do curso consistirá de:
	
	\begin{enumerate}
		\item Prova parcial (25\%).
		\item Prova final (35\%)
		\item Exercícios em monitoria (20\%).
		\item Trabalho (20\%).
	\end{enumerate}
	
	\paragraph{Bibliografia:} as notas de aula de curso, bem como os códigos apresentados em aula, serão 
\end{document}