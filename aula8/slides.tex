% !TeX document-id = {0be8c18c-9430-4e9a-bdd9-12beadebfebc}
% !TeX TXS-program:bibliography = txs:///biber
\documentclass[11pt]{beamer}

\usepackage[brazilian]{babel}

\uselanguage{portuguese}
\languagepath{portuguese}
\deftranslation[to=portuguese]{Theorem}{Teorema}
\deftranslation[to=portuguese]{theorem}{teorema}
\deftranslation[to=portuguese]{Example}{Exemplo}
\deftranslation[to=portuguese]{example}{exemplo}
\deftranslation[to=portuguese]{Lemma}{Lema}
\deftranslation[to=portuguese]{lemma}{Lema}
\deftranslation[to=portuguese]{Corollary}{Corolário}
\deftranslation[to=portuguese]{corollary}{corolário}
%\deftranslation[to=portuguese]{and}{e}


\usepackage[utf8]{inputenc}
\usepackage[T1]{fontenc}
\usepackage{lmodern}
\usepackage{amsmath}
\usepackage{amssymb}
\usepackage{mathtools}
\usepackage{color}
\usepackage{pgfplots}
\usepackage{tikz}
\usepackage{subcaption}
%\usepackage{appendixnumberbeamer}

\newenvironment{transitionframe}{
	\setbeamercolor{background canvas}{bg=yellow}
	\begin{frame}}{
	\end{frame}
}
\usetheme{default}
\usefonttheme{structuresmallcapsserif}

%% I use a beige off white for my background
\definecolor{MyBackground}{RGB}{255,253,218}
\useinnertheme[shadow]{rounded}
\setbeamercolor{block title}{bg=MyBackground}
\setbeamercolor{block body}{bg=MyBackground}
\setbeamercolor{example title}{bg=MyBackground}
\setbeamercolor{example body}{bg=MyBackground}


\newcommand{\blue}[1]{\textcolor{blue}{#1}}
\newcommand{\red}[1]{\textcolor{red}{#1}}
\newcommand{\purple}[1]{\textcolor{purple}{#1}}
\newcommand{\gray}[1]{\textcolor{gray}{#1}}
\setbeamertemplate{navigation symbols}{}
%\setbeamertemplate{page number in head/foot}[appendixframenumber]

%\usepackage{graphics}
\usepackage{graphicx}

\definecolor{blue_emph}{RGB}{0,114,178}
\definecolor{red}{RGB}{213,94,0}
\definecolor{yellow}{RGB}{240,228,66}
\definecolor{green}{RGB}{0,158,115}
\definecolor{purple}{RGB}{204,121,167}
\definecolor{orange}{RGB}{230,159,0}
\definecolor{lightblue}{RGB}{86,180,233}

%\setbeamercolor{frametitle}{fg=blue}
%\setbeamercolor{title}{fg=blue}
\setbeamertemplate{footline}[frame number]
\setbeamertemplate{navigation symbols}{} 
\setbeamertemplate{itemize items}{-}
%\setbeamercolor{itemize item}{fg=blue}
%\setbeamercolor{itemize subitem}{fg=blue}
\setbeamertemplate{enumerate items}[default]
%\setbeamercolor{enumerate subitem}{fg=blue}
\setbeamercolor{button}{bg=MyBackground,fg=blue}
\usefonttheme{structuresmallcapsserif}

%\setbeamercolor{section in toc}{fg=blue}
%\setbeamercolor{subsection in toc}{fg=red}
\setbeamersize{text margin left=1em,text margin right=1em} 


\usepackage{appendixnumberbeamer}

\usepackage[
backend=biber,
uniquename=false,
uniquelist=false,
style=authoryear,
natbib=true
]{biblatex}
\addbibresource{../bibliography.bib}

\newenvironment{wideitemize}{\itemize\addtolength{\itemsep}{10pt}}{\enditemize}
\newenvironment{wideenumerate}{\enumerate\addtolength{\itemsep}{10pt}}{\endenumerate}
\newenvironment{halfwideitemize}{\itemize\addtolength{\itemsep}{0.5em}}{\enditemize}
\newenvironment{halfwideenumerate}{\enumerate\addtolength{\itemsep}{0.5em}}{\endenumerate}


\author{Luis A. F. Alvarez}
\title{EAE1223: Econometria III}
\subtitle{Aula 8 - Modelos causais reduzidos}
%\logo{}
%\institute{}
\date{\today}
%\subject{}
%\setbeamercovered{transparent}

\begin{document}

\begin{frame}[plain]
	\maketitle
\end{frame}

\begin{frame}{Modelo}
	\begin{itemize}
		\item 	Seja $\{Y_t, \boldsymbol{X}_t\}_{t\in \mathbb{Z}}$ um processo vetorial de interesse, onde $Y_t$ é escalar e $\boldsymbol{X}_t$ é um vetor $d \times 1$.
		\item Pesquisador postula o seguinte modelo para $Y_t$:
		
		\begin{equation}
			\label{eq_structural}
			Y_t = \alpha + \beta ' \boldsymbol{X}_t + u_t\, ,
		\end{equation}
		onde $\beta$ é \textbf{definido} como o efeito causal (\textit{ceteris paribus}) de manipulações hipotéticas de cada uma das entradas de $\boldsymbol{X}$ sobre $Y$, e $u$ são os demais determinantes não observados do sistema.
		\item Perguntas desta aula:
		\begin{itemize}
			\item Sob quais condições estimador de MQO $\hat\beta$ estima consistentemente $\beta$?
			\begin{itemize}
				\item Consistência: para qualquer tolerância $\epsilon > 0$, $\lVert \hat \beta- \beta \rVert \leq \epsilon$ com alta probabilidade, para $T$ suficientemente grande.
			\end{itemize}
			\item Sob quais condições podemos usar a distribuição normal para fazer inferência sobre $\beta$, com base no estimador $\hat \beta$ e estatística $t$?
		\end{itemize}
	\end{itemize}

\end{frame}

\begin{frame}{Caso 1: $\{Y_t,\boldsymbol{X}_t\}$ é I(0)}
	\begin{itemize}
		\item Se o processo $\{Y_t,\boldsymbol{X}_t\}$ é I(0), estamos no mundo da Aula 1.
		\item Neste caso, estimador de MQO é consistente se:
		
		$$\operatorname{cov}(\boldsymbol{X}_t, u_t) = \boldsymbol{0} $$
		ou seja, não há relação sistemática entre determinantes observados e não observados.
		\item Nesse caso, podemos usar estatísticas $t$ e valores críticos normais para realizar inferência.
		\begin{itemize}
			\item No entanto, é apropriado usar {\color{blue}erros padrão HAC} para levar em conta heterocedasticidade e correlação serial em $u_t$.
			\item Esses erros padrão são válidos com $T$ grande (mesmo requerimento de consistência).
		\end{itemize}
	\end{itemize}
\end{frame}

\begin{frame}{Caso 2: $\{Y_t,\boldsymbol{X}_t\}$ é I(1)}
	\begin{itemize}
		\item Se o processo consiste de variáveis I(1), há risco de inferência espúria.
		\item No entanto, vimos na aula anterior as condições para que isso não ocorra, e $\hat \beta$ estime consistentemente $\beta$.
		\item Essas condições são:
		\begin{enumerate}
			\item $\{u_t\}_{t \in \mathbb{Z}}$ é estacionário, isto é, há relação de cointegração em $\{Y_t,\boldsymbol{X}_t\}$. {\color{blue}que envolve $Y_t$}.
			\item $\operatorname{cov}(\boldsymbol{X}_t, u_t)=\boldsymbol{0}, \quad \forall t$.
		\end{enumerate}
		\item Se as duas condições acima são satisfeitas, vimos que $\beta$ {\color{blue}tem a interpretação adicional de uma tendência de longo prazo}.
		\item Vimos que a Condição 1 é diretamente testável via procedimento de Engle-Granger.
		\begin{itemize}
			\item Condição 2 é intestável, de modo geral, sem hipóteses adicionais (assim como em cursos anteriores de Econometria).
		\end{itemize}
		\item Para podermos realizar inferência sobre $\beta$ usando erros padrão HAC e valores críticos normais, vimos na aula anterior que precisamos da hipótese adicional:
		\begin{enumerate}
			\item[3.] $\operatorname{cov}(\Delta \boldsymbol{X}_t, u_s) = 0, \quad  \forall t,s$. Essa condição é testável com base no correlograma cruzado de $\Delta \boldsymbol{X}_t$ e resíduos da regressão.
		\end{enumerate}
	\end{itemize}
	\end{frame}
	
	
	\begin{frame}{Caso 2: $\{Y_t,\boldsymbol{X}_t\}$ é I(1) (cont.)}
		\begin{itemize}
			\item Caso não haja cointegração envolvendo $Y_t$, sabemos que o estimador de MQO do modelo \eqref{eq_structural} leva a conclusões espúrias.
			\item Nesse caso, podemos considerar o estimador de MQO $\tilde \beta$ do modelo \eqref{eq_structural} em primeira diferença:
			
			$$\Delta Y_t = \beta' \Delta \boldsymbol{X}_t + \Delta u_t$$
			\item O estimador de MQO deste modelo será consistente sob a hipótese de identificação alternativa:
			$$\operatorname{cov}( \Delta \boldsymbol{X}_t, \Delta {u}_t) = \boldsymbol{0}\, ,$$
			isto é, variações nos determinantes não observáveis são não sistematicamente relacionadas a variações nos observáveis.
			\item Uma condição suficiente (intestável) para essa hipótese valer é que:
			$$\operatorname{cov}(\boldsymbol{X}_t, {u}_{s})=\boldsymbol{0}, \quad  \forall t, s \in\{t-1,t,t+1\}\, .$$
			\item Inferência nesse caso é convencional.
		\end{itemize}
	\end{frame}
	
	\begin{frame}{Caso 3: $\{Y_t,\boldsymbol{X}_t\}$ envolve variáveis I(0) e I(1)}
		\begin{itemize}
			\item Se $\{Y_t,\boldsymbol{X}_t\}$ envolvem uma mistura de processos I(0) e I(1), há algumas condições sob as quais é possível realizar inferência sobre um subconjunto dos parâmetros da forma convencional \citep{Sims1990}.
			\item No entanto, a maneira mais simples de lidar com a não estacionariedade é trabalhar com o modelo em diferenças:
				$$\Delta Y_t = \beta' \Delta \boldsymbol{X}_t + \Delta u_t\, ,$$
				e proceder como no slide anterior.
		\end{itemize}
		\end{frame}
		
			\begin{frame}{Caso 4: $\{Y_t,\boldsymbol{X}_t\}$ envolve variáveis I(0) e \textit{trend-stationary}}
				\begin{itemize}
					\item Se a única fonte de não estacionariedade no modelo é deterministíca, podemos realizar inferência com base em valores críticos normais.
					\item No entanto, é importante compreender que os efeitos estimados, nesse caso, serão \textbf{dominados} pela relação determinística entre as tendências de $Y_t$ e de $\boldsymbol{X}_t$.
					\begin{itemize}
						\item \textbf{Exemplo:} se $Y_t$ é \textit{trend-stationary} e $\boldsymbol{X}_t$ é I(0), $\hat \beta$ convergirá a  zero, visto que comportamento do processo $Y_t$ é {\color{blue}dominado} pela tendência.
					\end{itemize}
					\item Nesses casos, pode ser mais interessante considerar um modelo que postula relações causais para as variáveis \textit{detrended}, isto é, uma relação causal para os desvios da série em torno de suas tendências.
					\item Isso pode ser implementado fazendo o \textit{detrending} prévio das séries, ou incluindo explicitamente tendência no modelo linear causal:
					\begin{equation*}
						y_t = \alpha + \delta t + \gamma'\boldsymbol{X}_t+\xi_t \, ,
					\end{equation*}
					\item Estimador é consistente e inferência é válida se fatores não observados \textit{detrended} $\xi_t$ são não correlacionados com desvios de $\boldsymbol{X}_t$ de suas tendências.  
				\end{itemize}
			\end{frame}
			
			\begin{frame}{Incluindo defasagens}
				\begin{itemize}
					\item Em alguns casos, a hipótese de identificação pode ser mais crível se levarmos em conta a persistência de $Y_t$.
					\item Nesse caso, consideramos o modelo:
					
					$$Y_t = \alpha + \omega Y_{t-1} + \tau \boldsymbol{X}_{t}+ u_t\, ,$$
					\item Estimador de MQO será consistente se, crucialmente
$$\operatorname{cov}(\boldsymbol{X}_t, u_t) = \boldsymbol{0}$$
$$\operatorname{cov}(Y_{t-1}, u_t)=0$$
\item Segunda condição requer, no geral, que $\boldsymbol{X}_t$ não exerça efeito futuro sobre $u_t$, e que os $u_t$ sejam impredizíveis com base em seu passado, isto é:

$$\operatorname{cov}(\boldsymbol{X}_s, u_t)= \boldsymbol{0}, \quad \forall s \leq t$$
$$\operatorname{cov}(u_t,u_s)=0, \forall t,s$$
				\end{itemize}
			\end{frame}
\appendix
\begin{frame}[allowframebreaks]{Referências}
	\printbibliography
\end{frame}
\end{document}

