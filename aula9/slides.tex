% !TeX document-id = {0be8c18c-9430-4e9a-bdd9-12beadebfebc}
% !TeX TXS-program:bibliography = txs:///biber
\documentclass[11pt]{beamer}

\usepackage[brazilian]{babel}

\uselanguage{portuguese}
\languagepath{portuguese}
\deftranslation[to=portuguese]{Theorem}{Teorema}
\deftranslation[to=portuguese]{theorem}{teorema}
\deftranslation[to=portuguese]{Example}{Exemplo}
\deftranslation[to=portuguese]{example}{exemplo}
\deftranslation[to=portuguese]{Lemma}{Lema}
\deftranslation[to=portuguese]{lemma}{Lema}
\deftranslation[to=portuguese]{Corollary}{Corolário}
\deftranslation[to=portuguese]{corollary}{corolário}
%\deftranslation[to=portuguese]{and}{e}


\usepackage[utf8]{inputenc}
\usepackage[T1]{fontenc}
\usepackage{lmodern}
\usepackage{amsmath}
\usepackage{amssymb}
\usepackage{mathtools}
\usepackage{color}
\usepackage{pgfplots}
\usepackage{tikz}
\usepackage{subcaption}
%\usepackage{appendixnumberbeamer}

\newenvironment{transitionframe}{
	\setbeamercolor{background canvas}{bg=yellow}
	\begin{frame}}{
	\end{frame}
}
\usetheme{default}
\usefonttheme{structuresmallcapsserif}

%% I use a beige off white for my background
\definecolor{MyBackground}{RGB}{255,253,218}
\useinnertheme[shadow]{rounded}
\setbeamercolor{block title}{bg=MyBackground}
\setbeamercolor{block body}{bg=MyBackground}
\setbeamercolor{example title}{bg=MyBackground}
\setbeamercolor{example body}{bg=MyBackground}


\newcommand{\blue}[1]{\textcolor{blue}{#1}}
\newcommand{\red}[1]{\textcolor{red}{#1}}
\newcommand{\purple}[1]{\textcolor{purple}{#1}}
\newcommand{\gray}[1]{\textcolor{gray}{#1}}
\setbeamertemplate{navigation symbols}{}
%\setbeamertemplate{page number in head/foot}[appendixframenumber]

%\usepackage{graphics}
\usepackage{graphicx}

\definecolor{blue_emph}{RGB}{0,114,178}
\definecolor{red}{RGB}{213,94,0}
\definecolor{yellow}{RGB}{240,228,66}
\definecolor{green}{RGB}{0,158,115}
\definecolor{purple}{RGB}{204,121,167}
\definecolor{orange}{RGB}{230,159,0}
\definecolor{lightblue}{RGB}{86,180,233}

%\setbeamercolor{frametitle}{fg=blue}
%\setbeamercolor{title}{fg=blue}
\setbeamertemplate{footline}[frame number]
\setbeamertemplate{navigation symbols}{} 
\setbeamertemplate{itemize items}{-}
%\setbeamercolor{itemize item}{fg=blue}
%\setbeamercolor{itemize subitem}{fg=blue}
\setbeamertemplate{enumerate items}[default]
%\setbeamercolor{enumerate subitem}{fg=blue}
\setbeamercolor{button}{bg=MyBackground,fg=blue}
\usefonttheme{structuresmallcapsserif}

%\setbeamercolor{section in toc}{fg=blue}
%\setbeamercolor{subsection in toc}{fg=red}
\setbeamersize{text margin left=1em,text margin right=1em} 


\usepackage{appendixnumberbeamer}

\usepackage[
backend=biber,
uniquename=false,
uniquelist=false,
style=authoryear,
natbib=true
]{biblatex}
\addbibresource{../bibliography.bib}

\newenvironment{wideitemize}{\itemize\addtolength{\itemsep}{10pt}}{\enditemize}
\newenvironment{wideenumerate}{\enumerate\addtolength{\itemsep}{10pt}}{\endenumerate}
\newenvironment{halfwideitemize}{\itemize\addtolength{\itemsep}{0.5em}}{\enditemize}
\newenvironment{halfwideenumerate}{\enumerate\addtolength{\itemsep}{0.5em}}{\endenumerate}


\author{Luis A. F. Alvarez}
\title{EAE1223: Econometria III}
\subtitle{Aula 8 - Modelos vetoriais autorregressivos estruturais}
%\logo{}
%\institute{}
\date{\today}
%\subject{}
%\setbeamercovered{transparent}

\begin{document}

\begin{frame}[plain]
	\maketitle
\end{frame}

\begin{frame}{Um modelo para a descrição de uma economia}
	\begin{itemize}
		\item 	Seja $\{ \boldsymbol{Y}_t\}_{t\in \mathbb{Z}}$ um processo vetorial de interesse, onde $\boldsymbol{Y}_t$ consiste de $d$ variáveis econômicas, sobre as quais a teoria econômica têm algo a nos dizer sobre o comportamento conjunto.
		\item Um {\color{blue}modelo estrutural (causal) linear} para estas variáveis consiste em um sistema de $d$ equações da forma:
		\begin{equation} \label{eq_struct}
			\boldsymbol{A}_0 \boldsymbol{Y}_t = \boldsymbol{a} + \sum_{j=1}^p\boldsymbol{A}_j \boldsymbol{Y}_{t-j} + \boldsymbol{\epsilon}_t\, ,
		\end{equation}
		onde $\boldsymbol{A}_0$ é uma matriz $d \times d$ que explicita as relações contemporâneas (causais) entre as variáveis, e $\boldsymbol{\epsilon}_t$ é um ruído branco {\color{blue}contemporaneamente não correlacionado}, isto é $$ \mathbb{V}[\boldsymbol{\epsilon}_t] = \begin{bmatrix}
			\sigma_{1}^2 & 0 & \ldots & 0 \\
			0 & \sigma^2_2 & \ldots & 0 \\
			\vdots & \ldots & \ddots & \vdots \\
			0 & 0 & \ldots & \sigma^2_d
		\end{bmatrix}= \Omega_\epsilon $$
		\end{itemize}


\end{frame}

\begin{frame}{Choques econômicos fundamentais }
	\begin{itemize}
		\item A hipótese de que os erros de cada uma das equações são contemporaneamente não correlacionados supõe que o modelo que descreve a economia esteja {\color{blue}bem especificado}, de modo que o choque da $j$-ésima equação reflete a incerteza econômica fundamental associada a $Y_{jt}$.
		\begin{itemize}
			\item $\epsilon_{jt}$ reflete choques (surpresas ou inovações, não antecipadas com base no passado) nos determinantes essenciais de $Y_{jt}$, e não nos determinantes indiretos (via outras variáveis do sistema) de $Y_{jt}$. 
	
		\end{itemize}
				\item Esse tipo de hipótese é presumida em uma das perguntas clássicas da macroeconomia: quanto da flutuação econômica pode ser atribuída à política monetária vs fatores reais?
				\begin{itemize}
					\item Pergunta presume que existem inovações fundamentais, contemporaneamente ortogonais (não correlacionadas), em fatores monetários e reais, que permitem pensar nesta decomposição, visto que ela não faz sentido se os fatores fundamentais não fossem fundamentais (i.e. correlacionados).
					\item Veremos como nossa metodologia permite fazer explicitamente esta decomposição.
				\end{itemize}
	
	\end{itemize}
\end{frame}

\begin{frame}{Exemplo}
	\begin{itemize}
			\item Considere o comportamento conjunto de inflação ($\pi_t$), desemprego ($u_t$), expectativas de inflação ($\pi^e_t$) e taxa de juros nominal ($i_t$):
	\end{itemize}
			\begin{align}
		\pi_t = \sum_{j={\color{blue}1}}^p \theta_j \pi_{t-j} + \sum_{j={\color{red}0}}^p \beta_j (u_{t-j}-\bar{u}_{\text{neutro}}) + \sum_{j={\color{red}0}}^p \gamma_j \pi_{t-j}^e + \epsilon_{\pi,t} \tag{CP} \\
		u_t - \bar{u} = \sum_{j={\color{blue}1}}^p \omega_j (u_{t-j}-\bar{u}) + \sum_{j={\color{red}0}}^p \alpha_j (i_{t-j} - \pi^e_{t-j})  + \epsilon_{u,t} \tag{IS} \\  i_t = \bar{i} + \sum_{j={\color{blue}1}}^p \psi_j i_{t-j} + \sum_{j={\color{red}0}}^p \kappa_j (\pi_{t-j}-\pi_{\text{M}}) + \sum_{j={\color{red}0}}^p \phi_j (\pi_{t-j}^e-\pi_{\text{M}}) + \epsilon_{i,t}\tag{RM}\\
		\pi^e_t = \mu \pi_{M} +  \sum_{j={\color{blue}1}}^p \iota_j \pi^e_{t-j} + \theta_3 \sum_{j={\color{red}0}}^p (\nu_{1j} \pi_{t-j} + \nu_{2j} u_{t-j} + \nu_{3j} i_{t-j}) + \epsilon_{e,t} \tag{FE}
	\end{align}
	onde $\epsilon_{\pi,t}$ são choques de oferta (CP), $\epsilon_{u,t}$ choques de demanda (IS), $\epsilon_{i,t}$ são surpresas de política monetária (RM), e $\epsilon_{e,t}$ são ruídos na formação de expectativas.

\end{frame}

\begin{frame}{Representação autorregressiva do modelo linear estrutural}
	\begin{itemize}
		\item Se o sistema \eqref{eq_struct} oferece uma descrição {\color{blue}completa} da evolução de $\boldsymbol{Y}_t$, então, para uma dada trajetória pretérita $\{\boldsymbol{Y}_s: s \leq t-1\}$, e valores dos choques fundamentais $\boldsymbol{\epsilon}_t$, existe um único valor de $\boldsymbol{Y}_t$  que satisfaz \eqref{eq_struct}.
		\item Nesse caso, a matriz $\boldsymbol{A}_0$ admite inversa $\boldsymbol{B}= \boldsymbol{A}_0^{-1}$, e o sistema admite {\color{blue}representação autorregressiva}:
		
		\begin{equation}
			\label{eq_arriv}
			\boldsymbol{Y}_t = \underbrace{\boldsymbol{c}}_{= \boldsymbol{B}\boldsymbol{a}} + \sum_{j=1}^p \underbrace{\boldsymbol{C}_j}_{=\boldsymbol{B} \boldsymbol{A}_j} \boldsymbol{Y}_{t-j}  + \boldsymbol{B}\boldsymbol{\epsilon}_t\, .
		\end{equation}
		
		\item Modelo VAR em que o ruído branco $\boldsymbol{B}\boldsymbol{\epsilon}_t$ é uma combinação linear de choques fundamentais.
		\item Matriz $\boldsymbol{B}$ incorpora o efeito contemporâneo de inovações fundamentais sobre as variáveis em $\boldsymbol{Y}_t$. 
	\end{itemize}
\end{frame}

\begin{frame}{Modelo SVAR(p)}
\begin{itemize}
	\item 	Em diversas situações, não necessariamente queremos partir de \eqref{eq_struct} para chegar a \eqref{eq_arriv}
	\begin{itemize}
		\item Não necessariamente temos uma descrição completa da economia.
		\item De modo relacionado, a formulação \eqref{eq_arriv} pode ser compatível com mais de uma formulação estrutural linear completa.
	\end{itemize}
	\item Nesses casos, podemos definir diretamente {\color{blue}um modelo vetorial autorregressivo (semi)estrutural de ordem $p$}, SVAR(p), como o processo
	
		\begin{equation}
		\label{eq_svar}
		\boldsymbol{Y}_t = \boldsymbol{c}+ \sum_{j=1}^p \boldsymbol{C}_j \boldsymbol{Y}_{t-j} + \boldsymbol{B}\boldsymbol{\epsilon}_t\, ,
	\end{equation}
	onde $\boldsymbol{\epsilon}_t$ são inovações fundamentais, isto é ruídos brancos não contemporaneamente correlacionadas, e $\boldsymbol{B}$ é a matriz que captura os efeitos contemporâneos das inovações fundamentais sobre $\boldsymbol{Y}_t$.
\end{itemize}
\end{frame}

\begin{frame}{Função de resposta ao impulso}
	\begin{itemize}
		\item Os efeitos causais dinâmicos, na modelagem SVAR, são capturados pelos efeitos de surpresas nos choques fundamentais sobre o comportamento do sistema.
		\begin{itemize}
			\item Note que, por construção, $\boldsymbol{\epsilon}_t$ captura fatores não antecipados com base no passado. 
			\item Como $\boldsymbol{\epsilon}_t$ é não contemporaneamente correlacionado, faz sentido pensar em surpresas em um de seus componentes, mantidos os outros constantes.
	\end{itemize}
	\item Formalmente, o efeito causal de uma surpresa de uma unidade na $j$-ésima inovação do sistema em $t$, $\epsilon_{jt}$, sobre a $i$-ésima variável do sistema em $t+h$, $h\geq 0$, é dada pela {\color{blue}função de resposta ao impulso}
	$$F_{h}(i|j)  = \frac{\partial Y_{i,t+h}}{\partial \epsilon_{j,t}}$$
	\item Por exemplo, $F_{0}(i|j) = \boldsymbol{B}_{ij}$, $F_{1}(i|j) = \sum_{l=1}^d\boldsymbol{C}_{1, i,l} \boldsymbol{B}_{lj} = \sum_{l=1}^d\boldsymbol{C}_{1, i,l} F_{0}(l|j)$, $F_2(i|j) = \sum_{l=1}^d \boldsymbol{C}_{1, i,l} F_{1}(i|j) + \sum_{l=1}^d\boldsymbol{C}_{2, i,l} F_{0}(i|j)$, etc.
	\end{itemize}
\end{frame}

\begin{frame}{FRI normalizada, FRI acumulada}
	conteúdo...
\end{frame}


\begin{frame}{Decomposição da variância do erro de predição}
	conteúdo...
\end{frame}


\begin{frame}{SVAR(p) e VAR(p)}
	conteúdo...
\end{frame}

\begin{frame}{O problema de identificação causal no SVAR(p)}
	conteúdo...
\end{frame}
%\appendix
%\begin{frame}[allowframebreaks]{Referências}
%	\printbibliography
%\end{frame}
\end{document}

