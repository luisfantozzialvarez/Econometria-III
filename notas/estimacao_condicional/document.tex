% !TeX TXS-program:bibliography = txs:///biber
\documentclass[10pt,a4paper]{article}
\usepackage[T1]{fontenc}
\usepackage{graphicx}
\usepackage{mathtools}
\usepackage{amssymb}
\usepackage{amsthm}
\usepackage{thmtools}
\usepackage{xcolor}
\usepackage{nameref}
\usepackage{hyperref}
\usepackage{color}

\usepackage[
backend=biber,
style=authoryear,
natbib=true
]{biblatex}
\addbibresource{../bibliography.bib}


\title{\large EAE1223: Econometria III}
\author{\normalsize Estimação do ARMA(p,q) condicional}
\date{}
\begin{document}
\maketitle
\section*{Estimação do MA(1)}

Considere, de início o MA(1) sem intercepto:

$$Y_t = \epsilon_{t} + \theta_1 \epsilon_{t-1}$$

Se conhecêssemos $\epsilon_{0}$, poderíamos estimar o parâmetro $\theta_1$ como:
\begin{equation}
	\check{\theta}_1 \in \operatorname{argmin}_{c \in \mathbb{R}} \frac{1}{T}\sum_{t=1}^T (Y_t - c \check \epsilon_{t-1}(c))^2 \, ,
\end{equation}
onde $\check \epsilon_{0}(c) = \epsilon_{0}$, e, para $t > 0$, $\check \epsilon_{t}(c) = Y_t - c \check \epsilon_{t-1}(c)$.

O estimador $\check{\theta}_1$ não é factível, visto que desconhecemos $\epsilon_{0}$. Nós notamos, no entanto, que, dado um chute $\tilde{\epsilon}_0 = 0$ inicial para o ruído branco, somos capazes de construir o estimador alternativo

\begin{equation}
	\label{eq_feasible}
	\hat{\theta}_1 \in \operatorname{argmin}_{c \in \mathbb{R}} \frac{1}{T}\sum_{t=1}^T (Y_t - c \tilde \epsilon_{t-1}(c))^2 \, ,
\end{equation}
onde $\tilde{\epsilon}_0(c) = \tilde{\epsilon}_0$, e, recursivamente:

\begin{equation*}
	\begin{aligned}
		\tilde{\epsilon}_1(c) = Y_1 - c \tilde{\epsilon}_{0}(c) , \\ 
		\tilde{\epsilon}_2(c)= Y_2 - c\tilde{\epsilon}_{1}(c) \\
		\vdots \\
			\tilde{\epsilon}_T(c)=  Y_t - c\tilde{\epsilon}_{T-1}(c) 
	\end{aligned}
\end{equation*}

Note que o estimador factível efetivo $\hat \theta_1$ na prática não usa a primeira observação, visto que $(Y_1-c \tilde{\epsilon}_0(c))^2 = (Y_1)^2$.

No Apêndice A, nós mostramos que o chute inicial $\tilde{\epsilon}_0$ torna-se irrelevante, quando $T$ é grande, na região de invertibilidade do MA(1), $|c|<1$.

\paragraph{Obeservação:} note que a função objetivo $\frac{1}{T}\sum_{t=1}^T (Y_t - \tilde \epsilon_{t}(c) - c \tilde \epsilon_{t-1}(c))^2$ é trivialmente igual a zero, para qualquer valor de $c$. Logo, ela não pode ser usada na minimização. Devemos remover o ruído branco contemporâneo na estimação.

\section{Estimação condicional do ARMA(p,q)}
\appendix 
\section{Mostrando que chute inicial é irrelevante para $T$ grande, na região em que $|c|<1$}
Vamos mostrar agora que, para a região de parâmetros em que o MA é invertível, $|c|<1$, o chute inicial faz pouca diferença, com $T$ grande. De fato, observe que, para todo $t$:

$$\tilde{\epsilon}_t(c) = \check{\epsilon}_t(c) -c^t(\tilde{\epsilon}_0 - \epsilon_0)\, .$$

Usando esse fato, notamos que podemos escrever

\begin{equation}	
	\begin{aligned}
			\frac{1}{T}\sum_{t=1}^T (Y_t - c \tilde \epsilon_{t-1}(c))^2 = \frac{1}{T}\sum_{t=1}^T (Y_t - c \check \epsilon_{t-1}(c))^2 \\
		- \frac{2}{T}(\tilde{\epsilon}_0 - \epsilon_0)\sum_{t=1}^T(Y_t - c \check{\epsilon}_{t-1}(c))c^t 
		+\frac{1}{T}(\tilde{\epsilon}_0 - \epsilon_0)^2\sum_{t=1}^Tc^{2t}
	\end{aligned}
\end{equation}

Observe que, como $|c|<1$ temos que o terceiro termo é limitado por cima por:

$$\left|\frac{1}{T}(\tilde{\epsilon}_0 - \epsilon_0)^2\sum_{t=1}^Tc^{2t}\right| \leq \frac{1}{T}\frac{|c|^2}{1-|c|^2} |\tilde{\epsilon}_0 - \epsilon_0|^2 \, .$$

Assim, a contribuição desse termo desaparece com $T$ grande.

Quanto ao segundo termo, nós notamos que

$$(Y_t - c \check{\epsilon}_{t-1}(c)) = \sum_{i=1}^{t} (-c)^{t-i} Y_{i} + (-c^{t})\epsilon_0$$

Assim, podemos reescrever a parte relevante do segundo termo como:


$$	\frac{1}{T}\sum_{t=1}^T(Y_t - c \check{\epsilon}_{t-1}(c))c^t  =  \frac{1}{T}\sum_{t=1}^T c^{t}((-c)^{0} + (-c)^{1} + \ldots + (-c)^{T-t})Y_t +\frac{\epsilon_0}{T}\sum_{t=1}^T (-c^{2t})\, .$$

O segundo termo da expressão acima desaparece, com $T$ grande, quando $|c|<1$. Quanto ao outro termo, observe que ele possui média zero, visto que o MA(1) que estamos analisando possui média zero. Além disso:

\begin{equation}
	\begin{aligned}
		\mathbb{V}\left[ \frac{1}{T}\sum_{t=1}^T c^{t}((-c)^{0} + (-c)^{1} + \ldots + (-c)^{T-t})Y_t\right] = \\ \frac{1}{T^2} \sum_{t=1}^T c^{2t}((-c)^{0} + (-c)^{1} + \ldots +  (-c)^{T-t})^2 \mathbb{V}[Y_t] \\ + 2 \frac{1}{T^2}\sum_{i< j}c^{i+j} ((-c)^{0} + (-c)^{1} + \ldots +  (-c)^{T-i})((-c)^{0} + (-c)^{1} + \ldots +  (-c)^{T-j})\operatorname{cov}(Y_i,Y_j) = \\
		\frac{1}{T^2}  \mathbb{V}[Y_1] \sum_{t=1}^T c^{2t}((-c)^{0} + (-c)^{1} + \ldots +  (-c)^{T-t})^2 \\ + 2 \frac{1}{T^2}\operatorname{cov}(Y_1,Y_2) \sum_{i= 1}^{T-1}c^{2i+1} ((-c)^{0} + (-c)^{1} + \ldots +  (-c)^{T-i})((-c)^{0} + (-c)^{1} + \ldots +  (-c)^{T-i-1})
	\end{aligned}
\end{equation}
onde a última igualdade usa que o MA(1) é estacionário e que somente as autocovariâncias de primeira ordem são distintas de zero. Segue, então, que, quando	$T \to \infty$, a variância acima vai a zero, de onde concluímos que: 

$$\operatorname{plim}_{T\to \infty} \frac{2}{T}(\tilde{\epsilon}_0 - \epsilon_0)\sum_{t=1}^T(Y_t - c \check{\epsilon}_{t-1}(c))c^t = 0$$

Assim, temos que, para $|c| < 1$:

$$	\operatorname{plim}_{T\to \infty}\frac{1}{T}\sum_{t=1}^T (Y_t - c \tilde \epsilon_{t-1}(c))^2 =\operatorname{plim}_{T\to \infty} \frac{1}{T}\sum_{t=1}^T (Y_t - c \check \epsilon_{t-1}(c))^2\,,$$
de modo que a escolha de $\tilde{\epsilon}$ não afeta o comportamento do objetivo, para $T$ grande, na região invertível do MA.
\end{document}