% !TeX TXS-program:bibliography = txs:///biber
\documentclass[10pt,a4paper]{article}
\usepackage[T1]{fontenc}
\usepackage{graphicx}
\usepackage{mathtools}
\usepackage{amssymb}
\usepackage{amsthm}
\usepackage{thmtools}
\usepackage{xcolor}
\usepackage{nameref}
\usepackage{hyperref}
\usepackage{color}

\usepackage[
backend=biber,
style=authoryear,
natbib=true
]{biblatex}
\addbibresource{../bibliography.bib}


\title{\large EAE1223: Econometria III}
\author{\normalsize Exercícios sobre Processos Estocásticos}
\date{}
\begin{document}
	\maketitle
	\paragraph{Questão 1:}  Verifique se os seguintes processos ARMA(p,q) são estacionários e invertíveis, onde, no que segue, $\{u_t:t\in \mathbb{Z}\}$ é sempre um ruído branco. \texttt{Dica:} o comando \texttt{polyroot}, no R, calcula as raízes de um polinômio. O comando \texttt{abs} calcula o valor absoluto de um número complexo.
	
	\begin{enumerate}
		\item $y_t = 0.7 y_{t-2} + u_t$  ,
		\item $y_t = 0.5 y_{t-1} + 2y_{t-2} + u_t - 0.5 u_{t-1}$  ,
		\item $y_t = 1.5 y_{t-1}  - 0.5 y_{t-2} + u_t  - u_{t-1}$,
		\item $y_t = (1 - 2 L + 5 L^3)u_t$.
	\end{enumerate}
	
	\paragraph{Questão 2:} Seja $\{\epsilon_t\}_{t\in \mathbb{Z}}$ um ruído branco com variância igual a um. O processo a seguir é estacionário?
	
	$$(1-1.1L + 0.18L^2)y_t = \epsilon_t \, .$$
	
	Se sim, calcule sua variância e primeira autocovariância $\gamma_1 = \operatorname{cov}(y_t, y_{t-1})$, e proponha um algoritmo iterativo para calcular $\gamma_j = \operatorname{cov}(y_t, y_{t-j})$, $j>2$, como função das autocovariâncias anteriormente calculadas.
	
	
		\paragraph{Questão 3} Considere o processo:
	
	$$Y_t = (1+2.4L+0.8L^2)\epsilon_t\,,$$
	onde $\{\epsilon_t\}_t$ é ruído branco com variância unitária.
	
	O processo é estacionário? Se sim, calcule sua média, variância e função de autocovariância. Se não, justifique. O processo pode ser escrito como um AR($\infty$)? Justifique.

		\paragraph{Questão 4:} Considere o processo:
		
		$$y_t = \beta t + u_t \, ,$$
		onde $\{u_t\}$ é ruído branco, e $\beta > 0$.
		
		\begin{enumerate}
			\item Mostre que a média amostral de $\{y_t\}_t$, $\hat{\mu}_y = \frac{1}{T}\sum_{t=1}^T y_t$, diverge em probabilidade para $\infty$ quando o número de observações $T \to \infty$. \textit{Dica:} pela lei dos grandes números, $\operatorname{plim}_{T \to \infty}\frac{1}{T}\sum_{t=1}^T u_t  = 0$.
			\item Considere, agora, o estimador de MQO de $\beta$:
			
			$$\hat{\beta} = \frac{\sum_{t=1}^T t y_t}{\sum_{t=1}^T t^2}\, .$$
			
			Mostre que esse estimador satisfaz:
			
			$$\hat{\beta} = \beta +  \frac{\sum_{t=1}^T t u_t }{\sum_{t=1}^T t^2} \, ,$$
			e mostre que o segundo termo do lado direito da igualdade tem valor esperado zero.
			\item Mostre que 
			$$\lim_{T \to \infty}\mathbb{V}\left[ \frac{\sum_{t=1}^T t u_t }{\sum_{t=1}^T t^2}\right] = 0\, .$$
			\textit{Dica:} $\sum_{t=1}^T t^2 =\frac{1}{6} T(T+1)(2T+1)$. Como $\frac{\sum_{t=1}^T t u_t }{\sum_{t=1}^T t^2}$ tem média zero pelo item anterior e $\lim_{T \to \infty}\mathbb{V}\left[ \frac{\sum_{t=1}^T t u_t }{\sum_{t=1}^T t^2}\right] = 0$, nós concluiremos (não precisa demonstrar) que $\operatorname{plim}_{T\to \infty} \frac{\sum_{t=1}^T t u_t }{\sum_{t=1}^T t^2} = 0$. 
			
			\item Usando a conclusão anterior, mostre que $\operatorname{plim}_{T\to \infty} \hat{\beta} = \beta$.
			
		\end{enumerate} 
		
			\paragraph{Questão 5:} Vamos considerar dois passeios aleatórios \textbf{independentes}.
			$$y_t = y_{t-1} + u_t \, ,\quad  t=1,2,\ldots, $$
						$$x_t = x_{t-1} + v_t \, ,\quad  t=1,2\ldots, \, , $$
			com $y_0 = 0$ e $x_0 = 0$, e de tal forma que os processos $\{u_t\}_t$ e $\{v_t\}_t$ são \textbf{independentes} um do outro.
			
			\begin{enumerate}
				\item Simule os processos acima por cem períodos (por exemplo, usando as funções \texttt{ar.sim} ou \texttt{arima.sim} duas vezes, uma para cada processo). Esboce uma figura com a evolução dos dois processos no tempo.
				\item Usando as observações geradas, ajuste via MQO um modelo linear para prever $y_t$ como função de $x_t$ e um intercepto. Teste, a 5\% de significância, a hipótese nula de que o coeficiente associado a $x_t$ é zero. Qual é a conclusão do teste?
				\item Simule os dois processos novamente, mas agora por 500 períodos. Repita o procedimento do item anterior. Qual é a conclusão do teste?
						\item Simule os dois processos novamente, mas agora por 1000 períodos. Repita o procedimento do item (3). Qual é a conclusão do teste? Os resultados estão de acordo com o que você esperaria para a relação entre $y_t$ e $x_t$? Por quê?
			\end{enumerate}
			
\end{document}