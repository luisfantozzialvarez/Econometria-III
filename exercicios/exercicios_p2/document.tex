\documentclass[10pt,a4paper]{article}
\usepackage[T1]{fontenc}
\usepackage{amsmath,amssymb}
\begin{document}
	\begin{enumerate}
		\item Verifique se os seguintes processo vetoriais são estacionários ou cointegrados. No que segue, $\{u_{1,t},u_{2,t}\}_t$ é um ruído branco vetorial.
		
		\begin{enumerate}
			\item 		
			\begin{equation}
				\begin{aligned}
								y_t = 0.4 x_{t-1} + u_{1,t}\\
					x_t = 0.1 y_{t-1} + 0.6 x_{t-2} + u_{2,t}
					\end{aligned}
		\end{equation}
		\item 	\begin{equation}
			\begin{aligned}
				y_t = 4 x_{t-1} + u_{1,t}\\
				x_t = 0.1 y_{t-1} + 0.6 x_{t-2} + u_{2,t}
			\end{aligned}
		\end{equation}
	\end{enumerate}
	
	\item Considere o seguinte modelo para uma série de tempo:
	
	$$y_t = \sigma_t \nu_t$$
	$$\nu_t \overset{iid}{\sim}N(0,1)$$
	$$\sigma_t^2 = \omega_0 + \alpha_1 \epsilon_{t-1}^2\, , \quad \omega_0 > 0\, .$$
	
\begin{enumerate}
	\item Mostre que, se o processo $\{y_t\}$ é estacionário, então $|\alpha_1| < 1$.
	\item Mostre que, se o processo exibe curtose constante no tempo e finita, então $|\alpha_1| < \sqrt{1/3}$.
	\item Mostre que, na região   $|\alpha_1| < \sqrt{1/3}$, $y_t$ é leptocúrtico, isto é, sua curtose é maior que $3$.
\end{enumerate}
	\end{enumerate}
\end{document}