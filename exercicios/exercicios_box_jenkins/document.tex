% !TeX document-id = {1c0b4298-276a-4038-8e08-c4a1f4846da7}
% !TeX TXS-program:bibliography = txs:///biber
\documentclass[10pt,a4paper]{article}
\usepackage[T1]{fontenc}
\usepackage{graphicx}
\usepackage{mathtools}
\usepackage{amssymb}
\usepackage{amsthm}
\usepackage{thmtools}
\usepackage{xcolor}
\usepackage{nameref}
\usepackage{hyperref}
\usepackage{color}
\usepackage{float}

\usepackage[
backend=biber,
style=authoryear,
natbib=true
]{biblatex}
\addbibresource{../bibliography.bib}


\title{\large EAE1223: Econometria III}
\author{\normalsize Exercícios sobre a metodologia de Box-Jenkins}
\date{}
\begin{document}
	\maketitle
	\begin{enumerate}
\item Para cada uma das séries de tempo analisadas por vocês nos exercícios sobre raiz unitária.
\begin{enumerate}
	\item Divida o conjunto de dados em duas janelas, primeira em segunda, em que a primeira contém 4/5 dos períodos de tempo, e a segunda contém a quinta parte final. \textit{Dica:} use o comando \texttt{window} visto em aula.
	\item Com base na primeira janela, realize a etapa de identificação da metodologia de Box-Jenkins. Quais são os modelos candidatos? Por quê?
	\item Com base na primeira janela, estime os modelos candidatos. Com base nos critérios de diangóstico, selecione um ou mais modelos com boas métricas. Justifique suas escolhas.
	\item Compute as previsões para até um ano fora da primeira janela. Reporte os intervalos de predição associados. Como as predições se compararam ao que ocorreu na segunda janela? Qual é a interpretação do erro de previsão, nesse caso?
	\item Agora, para cada período na segunda janela, compute a previsão um passo à frente, com base nos dados até o período imediatamente anterior, para cada modelo Arima por você selecionado na primeira janela. Calcule o erro quadrático médio, um passo à frente, com base nos erros dessas previsões. Dentre os modelos por você estimados, qual se saiu melhor? Suas conclusões batem com o desempenho calculado no item anterior? Qual a diferença entre as métricas?
\end{enumerate}

		\item Dizemos que uma variável aleatória segue distribuição $\text{Laplace}(c,d)$ se admite densidade:

	$$f(x) = \frac{1}{2d_0}\exp(-|x-c|/d) \, .$$
	Nesse caso, a média da variável aleatória é $c$, e sua variância é $2d^2$.

	 Mostre que o estimador de máxima verossimilhança do parâmetro $\rho_0$ de um AR(1) (sem intercepto) com ruído branco $\epsilon_t \overset{\text{iid}}{\sim}\text{Laplace}(0,d_0)$, que maximiza a log-verossimilhança da distribuição de $Y_2,\ldots, Y_T$ condicional a $Y_1$, é idêntico ao estimador $\hat \rho$ que minimiza:
	$$\min_{b \in \mathbb{R}}\sum_{t=2}^T |Y_t - b Y_{t-1}|$$
\end{enumerate}

\end{document}