% !TeX TXS-program:bibliography = txs:///biber
\documentclass[10pt,a4paper]{article}
\usepackage[T1]{fontenc}
\usepackage{graphicx}
\usepackage{mathtools}
\usepackage{amssymb}
\usepackage{amsthm}
\usepackage{thmtools}
\usepackage{xcolor}
\usepackage{nameref}
\usepackage{hyperref}
\usepackage{color}
\usepackage{float}
\usepackage[
backend=biber,
style=authoryear,
natbib=true
]{biblatex}
\addbibresource{../bibliography.bib}


\title{\large EAE1223: Econometria III}
\author{\normalsize Exercícios sobre Raízes Unitárias}
\date{}
\begin{document}
	\maketitle
	\paragraph{Parte 1}\begin{enumerate}
		
		\item Dada uma série de tempo $Y_t$ com 230 observações, considere a estimação da seguinte especificação
		\begin{equation}
			\Delta Y_t = \alpha + \gamma \cdot Y_{t-1} + \sum_{s=1}^k \beta_k \cdot \Delta Y_{t-s} + u_t
		\end{equation}
		Na tabela abaixo, reportamos três quantidades: (1) o p-valor de $H_0: \gamma = 0$ contra $H_1: \gamma < 0$ baseado na estatística $\hat{t}$ e nos valores críticos tabulados por Dickey e Fuller; (2) a estatística $F$ do teste da nula conjunta $(\alpha, \gamma) = (0,0)$; (3) o p-valor de $H_0: \gamma = 0$ contra $H_1: \gamma < 0$ baseado na estatística $\hat{t}$ e em valores críticos normais.
		
		
		\begin{table}[H]
			\begin{center}
				\begin{tabular}{c|c}
					
					$p_{t,DF}$&0.8998    \\
					\hline
					$\hat{F}$&1.0292    \\
					\hline
					$p_{t,\text{Normal}}$&0.5606\\
				\end{tabular}
			\end{center}
		\end{table}
		No que segue, indique as conclusões do procedimento sequencial visto em aula, para um nível de significância de 10\%.
		\begin{enumerate}
			\item[(A)] Concluímos que a série \textbf{não} apresenta raiz unitária.
			\item[(B)] Concluímos que a série \textbf{apresenta} raiz unitária.
			\item[(C)] Concluímos que o modelo \textbf{não apresenta intercepto}. Nesse caso, devemos proceder à estimação do modelo sem intercepto, e realizar o teste $t$ baseado em valores críticos tabulados por Dickey e Fuller nesse modelo.
			\item[(D)] Nenhuma das alternativas anteriores.
		\end{enumerate}

	
	
	\item  Dada uma série de tempo $Y_{t}$ com 120 observações, considere a estimação da seguinte especificação:
	
	\begin{equation*}
		\Delta Y_{t}=\alpha+\beta t+\gamma \cdot Y_{t-1}+\sum_{s=1}^{k} \beta_{k} \cdot \Delta Y_{t-s}+u_{t} \, ,
	\end{equation*}
	onde o modelo é estimado através de mínimos quadrados ordinários, e $k$ é selecionado usando o critério MAIC.
	
	Na tabela abaixo, reportamos três quantidades: (1) o p-valor de $H_{0}: \gamma=0$ contra $H_{1}: \gamma<0$ baseado na estatística $\hat{t}$ e nos valores críticos tabulados por Dickey e Fuller; (2) a estatística $F$ do teste da nula conjunta $(\beta, \gamma)=(0,0)$; e (3) o p-valor de $H_{0}: \gamma=0$ contra $H_{1}: \gamma<0$ baseado na estatística $\hat{t}$ e em valores críticos normais.
	
	\begin{center}
		\begin{tabular}{c|c}
			$p_{t, D F}$ & 0.254 \\
			\hline
			$\hat{F}$ & 9.033 \\
			\hline
			$p_{t, \text { Normal }}$ & 0.035 \\
			\hline
		\end{tabular}
	\end{center}
	
	No que segue, indique qual das alternativas descreve as conclusões do procedimento sequencial visto em aula, para um nível de significância de $5 \%$. Justifique sua escolha.
	
	\begin{itemize}
		\item[(a)] Concluímos que a série não apresenta raiz unitária.
		
		\item[(b)] Concluímos que a série apresenta raiz unitária.
		
		\item[(c)] Concluímos que o modelo não apresenta tendência linear. Nesse caso, devemos proceder à estimação do modelo com intercepto e sem tendência linear, e realizar o teste $t$ baseado em valores críticos tabulados por Dickey e Fuller nesse modelo.
		
		\item[(d)] Nenhuma das alternativas anteriores.
	\end{itemize}
		\item Indique se as afirmações abaixo são verdadeiras ou falsas. Justifique sua resposta.
	
	\begin{itemize}
		\item[(a)] A metodologia de Elliott, Rothenberg e Stock provê um teste da nula de uma raiz unitária mais poderoso que o teste ADF, quando o modelo não apresenta componentes determinísticos.
		\item[(b)] A metodologia de Dickey-Pantulla para detectar se uma série é I(2), I(1) e I(0) consiste em, sequencialmente, rodar o teste ADF com a série original, e, se a nula não for rejeitada, rodamos novamente o teste ADF, agora testando a presença de raiz unitária na série em primeira diferença.
	\end{itemize}
\end{enumerate}
\paragraph{Parte 2}
Usando as duas séries que você analisou na lista de exercícios anterior.
\begin{enumerate}
	\item Para cada uma das séries, realize o procedimento sequencial descrito nos \textit{slides} para testar a presença de raiz unitária. Conduza os testes ao nível de significância de $10\%$. Qual a conclusão dos procedimentos?
	\item Para cada uma das séries, realize o teste de tendência determinística descrito nos \textit{slides}, já usando a série transformada obtida com base na conclusão dos testes no item anterior. Conduza os testes ao nível de significância de $10\%$.  Qual a conclusão de cada teste?
	
	\item Repita os dois itens anteriores a 5\% de significância. As conclusões mudaram?
\end{enumerate}
			
\end{document}