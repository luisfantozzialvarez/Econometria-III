% !TeX document-id = {1c0b4298-276a-4038-8e08-c4a1f4846da7}
% !TeX TXS-program:bibliography = txs:///biber
\documentclass[10pt,a4paper]{article}
\usepackage[T1]{fontenc}
\usepackage{graphicx}
\usepackage{mathtools}
\usepackage{amssymb}
\usepackage{amsthm}
\usepackage{thmtools}
\usepackage{xcolor}
\usepackage{nameref}
\usepackage{hyperref}
\usepackage{color}
\usepackage{float}

\usepackage[
backend=biber,
style=authoryear,
natbib=true
]{biblatex}
\addbibresource{../bibliography.bib}


\title{\large EAE1223: Econometria III}
\author{\normalsize Exercícios sobre cointegração}
\date{}
\begin{document}
	\maketitle
	
\paragraph{Questão 1} A subpasta \texttt{dados} contém séries econômicas para o IPCA (número índice, 100 = dezembro/1993), índice de preços aos consumidores nos Estados Unidos (número índice, 100= janeiro/1955) e taxa de câmbio nominal real-dólar.
	
	
	\begin{enumerate}
		\item Carregue o conjunto de dados. Restrinja-os a \textbf{janeiro de 2001} em diante. Rode os testes de não estacionariedade no \textbf{logaritmo} de cada uma das séries. Qual a conclusão dos testes? Cabe pensar em cointegração, neste caso? Por quê?
		
		\item Qual a conclusão da metodologia de Engle-Granger no conjunto de dados (em log)? Você pode fazer inferência no melhor preditor linear com base no estimador de MQO utilizado no teste? Por quê? Se a afirmativa for positiva, verifique se os coeficientes correspondem à paridade do poder de compra.
		
		\item Execute a metodologia de Johansen para o conjunto de dados. Qual é a conclusão da metodologia? Caso haja alguma relação de cointegração, teste se alguma das relações corresponde à paridade do poder de compra. \textit{Dica:} caso haja somente uma relação de cointegração, usar o comando \texttt{blrtest} visto em aula. Caso haja mais de uma, usar o comando \texttt{bh5lrtest} (veja \texttt{help(bh5lrtest)} para mais detalhes).
		
		\item Utilize o modelo multivariado apropriado para projetar a taxa de câmbio nos próximos doze meses.
\end{enumerate}

	\paragraph{Questão 2} Indique se as alternativas são verdadeiras ou falsas. Justifique sua resposta.

\begin{itemize}
	\item[(a)] Se rejeitamos a nula do teste de Portmanteau, então temos evidência de que pelo menos um dos coeficientes da última defasagem do VAR(p) é diferente de zero.
	\item[(b)] Se, num modelo VAR(p), uma variável $X$ não Granger causa uma variável $Y$, então a divulgação \textit{antecipada} do valor de $X$ para um período de tempo $T$ não gera ganhos preditivos passíveis de serem incorporados na atualização das projeções do modelo para $Y$ no período de tempo $T$, isto é, não há ganhos de realizar a atualização de \textit{nowcasting}.
	\item[(c)] Quando há três ou mais variáveis integradas de interesse, a metodologia de Engle-Granger é restritiva, no sentido de que a escolha de qual variável colocamos do lado esquerdo da equação envolve uma restrição sobre quem necessariamente deve aparecer na relação de cointegração, caso ela exista. 
	\item[(d)] Se rejeitamos a nula do procedimento de Engle-Granger, então as estimativas do modelo linear utilizado na metodologia não são espúrias, refletindo uma relação de longo prazo entre as variáveis.
	\item[(e)] O teorema de representação de Granger nos diz que, se um conjunto de variáveis integradas $\boldsymbol{X}_t$ pode ser representado por um VAR(p), então ele também admite uma representação de correção de erros, no sentido de que as variações de curto prazo $\Delta \boldsymbol{X}_t$ respondem a desvios das variáveis de suas relações de longo prazo.
	
	
\end{itemize}


\paragraph{Questão 3} A tabela a seguir nos dá as estatísticas de teste do máximo autovalor para um sistema integrado, bem como os valores críticos dos testes, a diferentes níveis de significância. Qual é a conclusão do procedimento de Johansen, a 5\% de significância, neste caso? E a 10\%? Tomando a conclusão do procedimento a 5\%, quais são as implicações do resultado para a previsão multivariada do sistema?

$$
\begin{array}{lrrrr}
	\hline \text { Rank } & \text { Test Statistic } & 10 \% & 5 \% & 1 \% \\
	\hline r<=4 & 5.19 & 6.50 & 8.18 & 11.65 \\
	r<=3 & 6.48 & 12.91 & 14.90 & 19.19 \\
	r<=2 & 17.59 & 18.90 & 21.07 & 25.75 \\
	r<=1 & 20.16 & 24.78 & 27.14 & 32.14 \\
	r=0 & 31.33 & 30.84 & 33.32 & 38.78 \\
	\hline
\end{array}
$$


\end{document}