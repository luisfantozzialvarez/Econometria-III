% !TeX TXS-program:bibliography = txs:///biber
\documentclass[10pt,a4paper]{article}
\usepackage[T1]{fontenc}
\usepackage{graphicx}
\usepackage{mathtools}
\usepackage{amssymb}
\usepackage{amsthm}
\usepackage{thmtools}
\usepackage{xcolor}
\usepackage{nameref}
\usepackage{hyperref}
\usepackage{color}

\usepackage[
backend=biber,
style=authoryear,
natbib=true
]{biblatex}
\addbibresource{../bibliography.bib}


\title{\large EAE1223: Econometria III}
\author{\normalsize Exercícios sobre Raízes Unitárias}
\date{}
\begin{document}
	\maketitle
	Usando as duas séries que você analisou no exercício anterior.
\begin{enumerate}
	\item Para cada uma das séries, realize o procedimento sequencial descrito nos \textit{slides} para testar a presença de raiz unitária. Conduza os testes ao nível de significância de $10\%$. Qual a conclusão dos procedimentos?
	\item Para cada uma das séries, realize o teste de tendência determinística descrito nos \textit{slides}, já usando a série transformada obtida com base na conclusão dos testes no item anterior. Conduza os testes ao nível de significância de $10\%$.  Qual a conclusão de cada teste?
	
	\item Repita os dois itens anteriores a 5\% de significância. As conclusões mudaram?
\end{enumerate}
			
\end{document}