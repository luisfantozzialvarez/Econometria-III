% !TeX document-id = {0be8c18c-9430-4e9a-bdd9-12beadebfebc}
% !TeX TXS-program:bibliography = txs:///biber
\documentclass[11pt]{beamer}
\uselanguage{portuguese}
\languagepath{portuguese}
\deftranslation[to=portuguese]{Theorem}{Teorema}
\deftranslation[to=portuguese]{theorem}{teorema}
\deftranslation[to=portuguese]{Example}{Exemplo}
\deftranslation[to=portuguese]{example}{exemplo}
\deftranslation[to=portuguese]{Lemma}{Lema}
\deftranslation[to=portuguese]{lemma}{Lema}
\deftranslation[to=portuguese]{Corollary}{Corolário}
\deftranslation[to=portuguese]{corollary}{corolário}
%\deftranslation[to=portuguese]{and}{e}

\usepackage[brazilian]{babel}
\usepackage[utf8]{inputenc}
\usepackage[T1]{fontenc}
\usepackage{lmodern}
\usepackage{amsmath}
\usepackage{amssymb}
\usepackage{mathtools}
\usepackage{color}
\usepackage{pgfplots}
\usepackage{tikz}

%\usepackage{appendixnumberbeamer}

\newenvironment{transitionframe}{
	\setbeamercolor{background canvas}{bg=yellow}
	\begin{frame}}{
	\end{frame}
}
\usetheme{default}
\usefonttheme{structuresmallcapsserif}

%% I use a beige off white for my background
\definecolor{MyBackground}{RGB}{255,253,218}
\useinnertheme[shadow]{rounded}
\setbeamercolor{block title}{bg=MyBackground}
\setbeamercolor{block body}{bg=MyBackground}
\setbeamercolor{example title}{bg=MyBackground}
\setbeamercolor{example body}{bg=MyBackground}


\newcommand{\blue}[1]{\textcolor{blue}{#1}}
\newcommand{\red}[1]{\textcolor{red}{#1}}
\newcommand{\purple}[1]{\textcolor{purple}{#1}}
\newcommand{\gray}[1]{\textcolor{gray}{#1}}
\setbeamertemplate{navigation symbols}{}
%\setbeamertemplate{page number in head/foot}[appendixframenumber]

%\usepackage{graphics}
\usepackage{graphicx}

\definecolor{blue_emph}{RGB}{0,114,178}
\definecolor{red}{RGB}{213,94,0}
\definecolor{yellow}{RGB}{240,228,66}
\definecolor{green}{RGB}{0,158,115}
\definecolor{purple}{RGB}{204,121,167}
\definecolor{orange}{RGB}{230,159,0}
\definecolor{lightblue}{RGB}{86,180,233}

%\setbeamercolor{frametitle}{fg=blue}
%\setbeamercolor{title}{fg=blue}
\setbeamertemplate{footline}[frame number]
\setbeamertemplate{navigation symbols}{} 
\setbeamertemplate{itemize items}{-}
%\setbeamercolor{itemize item}{fg=blue}
%\setbeamercolor{itemize subitem}{fg=blue}
\setbeamertemplate{enumerate items}[default]
%\setbeamercolor{enumerate subitem}{fg=blue}
\setbeamercolor{button}{bg=MyBackground,fg=blue}
\usefonttheme{structuresmallcapsserif}

%\setbeamercolor{section in toc}{fg=blue}
%\setbeamercolor{subsection in toc}{fg=red}
\setbeamersize{text margin left=1em,text margin right=1em} 


\usepackage{appendixnumberbeamer}

\usepackage[
backend=biber,
style=authoryear,
natbib=true
]{biblatex}
\addbibresource{../bibliography.bib}

\newenvironment{wideitemize}{\itemize\addtolength{\itemsep}{10pt}}{\enditemize}
\newenvironment{wideenumerate}{\enumerate\addtolength{\itemsep}{10pt}}{\endenumerate}
\newenvironment{halfwideitemize}{\itemize\addtolength{\itemsep}{0.5em}}{\enditemize}
\newenvironment{halfwideenumerate}{\enumerate\addtolength{\itemsep}{0.5em}}{\endenumerate}


\author{Luis A. F. Alvarez}
\title{EAE1223: Econometria III}
\subtitle{Aula 2 - Processos estocásticos}
%\logo{}
%\institute{}
\date{\today}
%\subject{}
%\setbeamercovered{transparent}

\begin{document}

	\begin{frame}[plain]
	\maketitle
	\end{frame}
	\begin{frame}{Espaço de probabilidade}
		\begin{itemize}
			\item Formalmente, o conceito utilizado para se definir a noção de incerteza associada a um problema é o de {\color{blue}espaço de probabilidade}.
			\item Um espaço de probabilidade é uma tripla $(\Omega, \Sigma, \mathbb{P})$, onde:
			\begin{itemize}
				\item $\Omega$ é um conjunto, denominado {\color{blue}espaço amostral}, contendo todos as possíveis realizações da incerteza.
				\item $\Sigma$ é uma coleção de subconjuntos de $\Omega$, denominada {\color{blue}$\sigma$-álgebra}. A cada subconjunto de $\Omega$ pertencente a $\Sigma$  damos o nome de {\color{blue}evento}. Os elementos de $\Sigma$ são aqueles para os quais somos capazes de definir a incerteza.
				\item uma {\color{blue}lei de probabilidade} $\mathbb{P}$ que atribui, a cada conjunto $E \in \Sigma$, um número $\mathbb{P}[E]$ entre $0$ e $1$. A lei de probabilidade satisfaz os {\color{blue}axiomas de Kolmogorov}.
			\end{itemize}
			\item Por que não definimos a probabilidade para todo subconjunto de $\Omega$?
			\begin{halfwideitemize}
				\item \textbf{Resposta:} se $\Omega$ é ``complexo'' (por exemplo, $[0,1]$), é impossível definir uma probabilidade que satisfaça todos os axiomas de Kolmogorov para todo subconjunto do espaço.
			\end{halfwideitemize}
		\end{itemize}
	\end{frame}
	
	\begin{frame}{Exemplo}
		\begin{halfwideitemize}
			\item Considere um lançamento de um dado não viciado.
			\item Nesse caso, espaço amostral é $\Omega = \{1,2,3,4,5,6\}$.
			\item Como lançamento é não viciado, sabemos que:
			$$\mathbb{P}[\{1\}]= \mathbb{P}[\{2\}] = \mathbb{P}[\{3\}] = \mathbb{P}[\{4\}] = \mathbb{P}[\{5\}] = \mathbb{P}[\{6\}] = 1/6 \,.$$
			\item Pelos axiomas da probabilidade, segue que podemos tomar $\Sigma$ como o conjunto de todos os subconjuntos de $\Omega$, e, para qualquer $E\subset \Sigma$:
			$$\mathbb{P}[E]= \mathbb{P}[\cup_{e \in E}\{e\}] = \sum_{e \in E}\mathbb{P}[\{e\}] = \frac{\# E}{6}\, ,$$
			onde $\# E$ é o número de elementos de $E$.
			\item Exemplo: probabilidade de que o lançamento de um número par é:
			$$\mathbb{P}[\{2,4,6\}]= \frac{3}{6} = \frac{1}{2}$$
		\end{halfwideitemize}
	\end{frame}
	
	
	\begin{frame}{Variável aleatória e processo estocástico}
		\begin{halfwideitemize}
			\item Uma {\color{blue}variável aleatória} $Z$ é uma {\color{blue}função}, com domínio no espaço amostral (onde definimos a incerteza), e valores em outro espaço (para nossos fins, os reais).
			\begin{itemize}
				\item Por exemplo, $(\Omega, \Sigma, \mathbb{P})$ um espaço de probabilidade descrevendo a incerteza associada aos retornos de ativos financeiros, e $Z: \Omega \mapsto \mathbb{R}$ é a variável aleatória que representa o retorno de um fundo.
				\item Incerteza em $(\Omega, \Sigma, \mathbb{P})$ traduz-se em incerteza em $Z$, i.e. $Z$ é incerto pois o valor $\omega \in \Omega$ que ocorre é incerto.
			\end{itemize}
			\item Um {\color{blue} processo estocástico} é uma coleção de variáveis aleatórias $\{X_t: t \in \mathcal{T}\}$, com domínio no \textbf{mesmo} espaço de probabilidade e indexada por um conjunto $\mathcal{T}$
			\item Uma {\color{blue}série de tempo} é um processo estocástico indexado no tempo, i.e. $\mathcal{T}$ é um conjunto de períodos.
			\begin{halfwideitemize}
				\item Como  tomaremos $\mathcal{T} = \mathbb{Z}$ ou $\mathcal{T}  =\mathbb{N}$.
				\item Para cada $\omega \in \Omega$, $\{ X_t(\omega): t \in \mathcal{T} \}$ é uma possível trajetória da série de tempo. Para cada $t \in \mathcal{T}$, $X_t$ é uma variável aleatória.
			\end{halfwideitemize}
		\end{halfwideitemize}
	\end{frame}
	
	\begin{frame}{Série de tempo estritamente estacionária}
		\begin{itemize}
			\item Uma série de tempo $\{X_t: t \in \mathcal{T}\}$ é dita estritamente estacionária se, para todo $t \in \mathcal{T}$, $j\in \mathbb{N}$:
			
			$$(X_t, X_{t+1}, \ldots, X_{t+j}) \overset{d}{=} (X_{t+h}, X_{t+1+h}, \ldots, X_{t+j+h})\,, \quad \forall h \geq 0 \, , $$
			onde $\overset{d}{=}$ significa igualdade das distribuições conjuntas, i.e. ?
			
			\begin{align*}
				\mathbb{P}[X_t \leq c_1, X_{t+1} \leq c_2, \ldots, X_{t+j} \leq c_j ] = \\ \mathbb{P}[X_{t+h} \leq c_1, X_{t+1+h} \leq c_2, \ldots, X_{t+j+h} \leq c_j ], \quad \forall c_1,c_2\ldots, c_j \, .
			\end{align*}
			
			\item Estacionariedade estrita requer que distribuição de qualquer número finito de períodos do processo seja a mesma ao longo do tempo.
		\end{itemize}
	\end{frame}
	
		\begin{frame}{Série de tempo fracamente estacionária}

	\end{frame}
	
	\begin{frame}{Função de autocovariância}
		conteúdo...
	\end{frame}
	
	\begin{frame}{Ruído branco}
		conteúdo...
	\end{frame}
	
	
	\begin{frame}{Ruído branco (ilustração)}
		conteúdo...
	\end{frame}
	
	
	\begin{frame}{Ruído branco (ilustração)}
		conteúdo...
	\end{frame}
	
		\begin{frame}{Processo MA(q)}
		conteúdo...
	\end{frame}
	
	\begin{frame}{Processo MA(q) (ilustração)}
		conteúdo...
	\end{frame}
	
		
	\begin{frame}{Processo AR(1) estacionário}
		conteúdo...
	\end{frame}
	
	\begin{frame}{Processo AR(1) estacionário (ilustração)}
		conteúdo...
	\end{frame}
	
		\begin{frame}{Processo AR(p) estacionário}
		conteúdo...
	\end{frame}
	
	
	\begin{frame}{Operador defasagem}
		conteúdo...
	\end{frame}
	
	\begin{frame}{Estacionariedade do AR(p)}
		\end{frame}
	%\begin{frame}[allowframebreaks]{Bibliografia}
	%\printbibliography
	%\end{frame}
\end{document}